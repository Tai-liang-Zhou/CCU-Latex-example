The latest trend that is catching attention of the majority of the tech industry is chatbots. Based on artificial intelligence (AI), a chatbot can learn how to react to different situations and communicate with the human in natural language. In addition, thanks for the various cloud platforms, chatbot can easy to deploy. As a result, chatbot applications can be seen tremendously on messaging applications on websites and mobile phones. Undoubtedly, a chatbot is a new way of interaction between humans and machines. However, a typical chatbot can also act as a simple question answering system that responses with formulated answers. Traditional conversational chatbots usually adopt a retrieved-based model. Developers have to provide a large amount of conversational data and classify those data to different intents. To avoid cumbersome development processes, we propose a method to build a chatbot by a sentence generation model which generates sequence sentences based on the generative adversarial network. The architecture of our model contains a generator that generates a diverse sentence with a corresponding action code, a dialogue manage code, and a persona code, and a discriminator that judges the sentences between the generated and the raw data. In the generator, we combine the attention model that responses for tracking conversational states with the sequence-to-sequence model using hierarchical long-short term memory to extract sentence information. For the discriminator, we calculate three types of rewards to assign low rewards for repeated sentences and high rewards for diverse sentences. Extensive experiments are presented to demonstrate the utility of our model which generates more diverse and information-rich sentences than those of the existing approaches.
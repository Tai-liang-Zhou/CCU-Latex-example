最近有個趨勢吸引大多數科技行業注意,那就是聊天機器人。基於人工智能(AI),聊天機器人可以學習如何應對不同的情況,並與人類用自然語言交流。此外,多虧有各種雲端平台讓聊天機器人可以很容易地部署,聊天機器人應用程式可以在訊息、手機、網頁或手機等上看到。

毫無疑問,聊天機器人是人類與機器互動的一種新方式。然而,一個典型的聊天機器人也代表了一個簡單的問題回答系統,公式化的回答。
傳統的對話聊天機器人通常采用基於檢索的模型。
開發人員必須提供大量的對話數據,並根據不同的任務對這些數據進行分類。
為了避免繁瑣的開發過程,我們提出了一種基於生成對抗性網絡生成句子的生成模型來構建聊天機器人。
我們模型的體系結構包含一個生成器,該生成器依照不同的編碼生成不同的句子、一個對話管理碼和一個人物碼,還有一個判別器,用於判斷生成的句子和原始數據。
在生成器中,我們結合注意模型追蹤句子狀態與使用雙向長短期記憶序列模型來提取句子信息的
對於判別器,我們計算了三種類型的回饋分數,分別為重覆句的低獎勵和不同句的高獎勵。
通過大量的實驗驗證了該模型的有效性,與現有的方法相比,該模型生成的句子更加多樣化,信息量也更加豐富。
\\
\\
\\
\textbf{關鍵字:}Deep Learning, Deep Learning, Chatbots, Service Robots, Natural Language Processing, Sentence Generation, Dialog Management.
\\
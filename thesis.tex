% 本latex 論文格式由DM+ Lab 修正

% 欲修改論文題目、作者、日期等等,請至同目錄之 ntuvars.tex 修改。

% 其他論文內容請至對應之 input 檔修改。

%其餘額外論文所需之文件尚有: (口試時網路上列印)
%  1. 學位考試同意書
%  2. 學位論文考試審定書
%  3. 博碩士論文授權書
%  4. 碩博士論文電子檔案上網授權書

% 論文封背附有doc檔案,在列印論文時請額外更改姓名、標題後於影印店列印實體論文時使用。

% 欲加入 Watermark 請取消 Watermark Settings 的註解file:///C:/Users/PhillipLab/Desktop/NTU_master_thesis/ntuvars.tex
% 並在需要 Watermark 的頁面加上 \AddToShipoutPicture{\WatermarkPicture} 
% 註:之後每一頁都會有 Watermark
% 在不需要 Watermark 的頁面加上 \ClearShipoutPicture,此後每一頁都將無 Watermark

\documentclass[oneside, a4paper, 12pt]{book}

\usepackage{amssymb}
\usepackage{times}
\usepackage{verbatim}
\usepackage{color}
\usepackage{url}
\usepackage{graphicx}
\usepackage{epsfig}
\usepackage{epstopdf}
\usepackage{array}
\usepackage{setspace}

% Watermark
\usepackage{eso-pic}

% Watermark Settings
\newcommand\WatermarkPicture{
    \put(0,0){
	    \parbox[b][\paperheight]{\paperwidth}{
            \vfill
            \centering	\includegraphics[width=200pt,keepaspectratio]{ccuwatermark}%
	        \vfill
        }
    }
}

% uncomment this if you want to indent the first paragraph
\usepackage{indentfirst}

\usepackage{ntu}

% Tables (http://en.wikibooks.org/wiki/LaTeX/Tables)
\usepackage{tabularx}
\usepackage{multirow}

% Mathematics (http://en.wikibooks.org/wiki/LaTeX/Mathematics)
\usepackage{amsmath}

% Algorithms (http://en.wikibooks.org/wiki/LaTeX/Algorithms)
\usepackage[ruled]{algorithm2e}
\usepackage{algpseudocode}
\renewcommand{\algorithmicrequire}{\textbf{Input:}}
\renewcommand{\algorithmicensure}{\textbf{Output:}}

% Theorems / Definition (http://en.wikibooks.org/wiki/LaTeX/Theorems)
\usepackage{amsthm}
\theoremstyle{definition}
\newtheorem{definition}{Definition}
\newcommand{\etal}{\emph{et al. }}
\newcommand{\eg}{\emph{e.g.}}
\newcommand{\ie}{\emph{i.e.}}
\newcommand{\etc}{\emph{etc.}}
% \usepackage{graphicx} 
%\usepackage{bmpsize}

%%%% add by Willy  may not be OK
\usepackage[nolist,printonlyused]{acronym}
\usepackage{nameref}
\usepackage{mathtools} 
\usepackage[table,xcdraw]{xcolor}
\usepackage{caption}
\usepackage[caption=false]{subfig}
\newtheorem{Definition}{Definition}
\usepackage{lscape} %table rotate
\usepackage{verbatim}
%\usepackage{subcaption} 
\DeclarePairedDelimiter{\ceil}{\lceil}{\rceil}
%%%%

\setcounter{tocdepth}{2}

\pagestyle{plain}

\begin{document}
% side page, used for printing on spline.
% 若需要調整,請修改 ntu.sty 中的 \newcommand{\makeside}{}。
\makeside

\doublespacing
% cover page
\maketitle

\begin{CJK}{UTF8}{nkai}
\CJKhorz
 %蝴蝶頁
\thispagestyle{empty}
\newpage
\end{CJK}

% 論文中的第一頁,且需要加上浮水印。
\AddToShipoutPicture{\WatermarkPicture}
\maketitle
%\ClearShipoutPicture %控制每頁浮水印

\frontmatter

\begin{CJK}{UTF8}{nkai}
\CJKhorz

% 論文中的第二、三、四頁,請依順序插入學位考試同意書、學位論文考試審定書、博碩士論文授權書。
\makecertification

% comment one of the following unless you are sure you want to 
% have both english and chinese acknowledgements in your thesis
%\begin{dedicationEN}
%\doublespacing
%	%\input{dedicationEN}
%\end{dedicationEN}

%\begin{acknowledgementsEN}
%\doublespacing
%	%\input{acknowledgementsEN}
%\end{acknowledgementsEN}

% 論文中的第五頁為致謝。(請更改內容)
\begin{acknowledgementsCH}
\doublespacing
感謝心得分享
\end{acknowledgementsCH}

% 論文中的第六、七頁分別為中文摘要與英文摘要。(請更改內容)
\begin{abstractCH}
\doublespacing
最近有個趨勢吸引大多數科技行業注意,那就是聊天機器人。基於人工智能(AI),聊天機器人可以學習如何應對不同的情況,並與人類用自然語言交流。此外,多虧有各種雲端平台讓聊天機器人可以很容易地部署,聊天機器人應用程式可以在訊息、手機、網頁或手機等上看到。

毫無疑問,聊天機器人是人類與機器互動的一種新方式。然而,一個典型的聊天機器人也代表了一個簡單的問題回答系統,公式化的回答。
傳統的對話聊天機器人通常采用基於檢索的模型。
開發人員必須提供大量的對話數據,並根據不同的任務對這些數據進行分類。
為了避免繁瑣的開發過程,我們提出了一種基於生成對抗性網絡生成句子的生成模型來構建聊天機器人。
我們模型的體系結構包含一個生成器,該生成器依照不同的編碼生成不同的句子、一個對話管理碼和一個人物碼,還有一個判別器,用於判斷生成的句子和原始數據。
在生成器中,我們結合注意模型追蹤句子狀態與使用雙向長短期記憶序列模型來提取句子信息的
對於判別器,我們計算了三種類型的回饋分數,分別為重覆句的低獎勵和不同句的高獎勵。
通過大量的實驗驗證了該模型的有效性,與現有的方法相比,該模型生成的句子更加多樣化,信息量也更加豐富。
\\
\\
\\
\textbf{關鍵字:}Deep Learning, Deep Learning, Chatbots, Service Robots, Natural Language Processing, Sentence Generation, Dialog Management.
\\x
\end{abstractCH}

\begin{abstractEN}
\doublespacing
The latest trend that is catching attention of the majority of the tech industry is chatbots. Based on artificial intelligence (AI), a chatbot can learn how to react to different situations and communicate with the human in natural language. In addition, thanks for the various cloud platforms, chatbot can easy to deploy. As a result, chatbot applications can be seen tremendously on messaging applications on websites and mobile phones. Undoubtedly, a chatbot is a new way of interaction between humans and machines. However, a typical chatbot can also act as a simple question answering system that responses with formulated answers. Traditional conversational chatbots usually adopt a retrieved-based model. Developers have to provide a large amount of conversational data and classify those data to different intents. To avoid cumbersome development processes, we propose a method to build a chatbot by a sentence generation model which generates sequence sentences based on the generative adversarial network. The architecture of our model contains a generator that generates a diverse sentence with a corresponding action code, a dialogue manage code, and a persona code, and a discriminator that judges the sentences between the generated and the raw data. In the generator, we combine the attention model that responses for tracking conversational states with the sequence-to-sequence model using hierarchical long-short term memory to extract sentence information. For the discriminator, we calculate three types of rewards to assign low rewards for repeated sentences and high rewards for diverse sentences. Extensive experiments are presented to demonstrate the utility of our model which generates more diverse and information-rich sentences than those of the existing approaches.
\end{abstractEN}

\tableofcontents
\listoffigures
\listoftables

\end{CJK}

\mainmatter

%%主文(請更改內容)
% input your thesis content here
%%%%%%%%%%%%%%%%%%%%%%%%%%%%%%%%%%%%%%%%%%%%%%%%%%%%%%%%%%
%Chapter 1: Introduction
\begin{CJK}{UTF8}{nkai}
\CJKhorz
\chapter{Introduction}\label{ch-introduction}
% 英文
\section{English}
Test add English

% 中文
\section{Chinese}
新增中文

% 圖片
\section{Picture}
add picture \ref{fig:SystemDesign}

\begin{figure}[h!]
\centerline{\psfig{figure=figures/FBChatbot.pdf, width=\linewidth}}
\caption{The architecture of Language model.}
\label{fig:SystemDesign}
\end{figure}

% 表單 , 
\section{Table}
create table \ref{tab:my-table}

\begin{table}[]
\centering
\caption{first table}
\label{tab:my-table}
\begin{tabular}{lll}
\hline
test1 & test2 & test3 \\ \hline
a     & b     & c     \\ \hline
x     & y     & z     \\ \hline
\end{tabular}
\end{table}


% 公式 , https://mathpix.com/
\section{Formula}
this is cite example \ref{eq:Formula}

\begin{equation}\label{eq:Formula}
    H(Y | X)=\sum_{x \in \mathcal{X}, y \in \mathcal{Y}} p(x, y) \log \left(\frac{p(x)}{p(x, y)}\right)
\end{equation}\label{eq:Formula}

% 引用
\section{Cite bib}
this is cite example \cite{Bi-LSTM}

% etal、eg、ie、etc
\section{etal eg ie etc}
\etal, \eg, \ie, \etc
\end{CJK}


% 据预处理

% 在神经网络中,对于文本的数据预处理无非是将文本转化为模型可理解的数字,这里都比较熟悉,不作过多解释。但在这里我们需要加入以下四种字符,<PAD>主要用来进行字符补全,<EOS>和<GO>都是用在Decoder端的序列中,告诉解码器句子的起始与结束,<UNK>则用来替代一些未出现过的词或者低频词。

% < PAD>: 补全字符。
% < EOS>: 解码器端的句子结束标识符。
% < UNK>: 低频词或者一些未遇到过的词等。
% < GO>: 解码器端的句子起始标识符。

% %%%%%%%%%%%%%%%%%%%%%%%%%%%%%%%%%%%%%%%%%%%%%%%%%%%%%%%%%%
% %Chapter 4: Your work
% \chapter{Data Pre-processing} \label{ch-preprocessing}
% % off-line part : data preprocessing, historical model construction and graph construction
In this section, we describe the data pre-processing component which performs the data cleansing, data transformation, and sensor road network construction.

%	Data preprocessing 
%		why? how?
%		for sensor meta data 
%			lack of absPM and (latitude,longitude) 
%		for data records 
%			returned by type ML 
\section{Data Cleansing} \label{subsubsec:dc} 

%TODO Introduce the data set first, before you describe how to process it. ~Yuling
We use the real-world traffic dataset called Caltrans Performance Measurement System (PeMS)~\cite{pemsdataset} to validate our system. The sensors are deployed on the freeway system of the state of California to collect traffic data in real-time. 
PeMS collects data from various types of vehicle detector stations, including inductive loops, side-fire radars, and magnetometers. A machine installed along roads in California records data from loops on the freeway. Detectors sense the number of vehicles crossed the loops as traffic flow, and the average fraction of time that vehicles travel on the loops as road occupancy. 
The records which include traffic flow, speed, occupancy, direction, and so on are collected and aggregated every five minutes. 
%TODO  mention why you need the pre-processing 
The sensors collect traffic data all year round, and it is inevitable that there will be some wrong records in some fields or missing records at some time intervals on any road segments. In order to make a precise estimation, we have to deal with these noisy data without affecting the performance of our model. 

For the data cleansing procedure, we handle the issue of missing data fields in the data records fetched from the sensors. In this work, we use the real sensor data, the PeMS dataset, to validate our system. Based on our observation from the real sensor dataset, some sensors either lack spatial information or contain incomplete returned records which have missing data fields. 
% TODO Why does removing Meta data help?
For the issue of missing spatial information, we remove the sensor meta data if the absolute post-mile fields, latitude, and longitude fields are blank. Due to the importance of these spatial characteristics, we have to remove these sensors which lack these fields to help us construct a precise sensor road network. For the issue of incomplete returned records, only two types of sensors return the actual traffic flow at a five-minute period. One is the ML type, which represents a main lane; the other is the HV type, which represents the high-occupancy lane. Since the HV type sensor connectivity information necessary for the sensor graph construction ($\eg$, entrance and connection between the ML type sensors) is not available, and 96 $\%$ of HV type sensors have only one lane, we ignore HV type sensors and only use the ML type sensors to detect the major causes on the main line.


% Historical model contruction definition day-of-week model why? people have
% different patterns on different day of week example and figure
\section{Data Transformation}\label{subsubsec:hm} 
We utilize the historical data collected in the past and transform it into a day-of-the-week format, which is used to detect major causes of abnormal events. The normal and abnormal patterns have been trained by learning from the transformed data. To a certain extent, the mean and standard deviation of the historical data may reflect the normal condition. For example, the mean of the traffic flow in the time interval from 8:30 to 9:00 is different from the mean in the time interval from 22:30 to 23:00. As in the example shown in Table~\ref{tab:day-of-week_Model}, we calculate pairs of the mean and standard deviation of traffic flow collected by sensors in 48 time intervals for each day of the week. The duration of every time interval is 30 minutes in this example.
We consider that the temporal patterns retrieved from the sensor data can be found in a one week period, and we store the mean and standard deviation values in a day-of-week table for each sensor over two months. As an example in Table ~\ref{tab:day-of-week_Model}, (156.00),(5.88) is a number pair which contains mean and standard deviation values of records collected by Sensor 1 at the time interval {[0:00, 0:30)} on Mondays over two months in the training dataset.



\begin{table}[h]
\begin{center}
\caption{An example of the day-of-week model.}
\label{tab:day-of-week_Model}
 \resizebox{\linewidth}{!}{%Resizes the table to \textwidth
\begin{tabular}{|c|c|c|c|c|c|}
\hline
Time interval & Mon              & Tue              & \ldots & Sat              & Sun              \\ \hline
\multicolumn{6}{|c|}{Sensor 1}                                                                     \\ \hline
0:00          & (172.00,3.26) & (88.33,1.24)   & \ldots & (88.00,0.81)   & (97.00,1.63)   \\ \hline
0:30          & (156.00,5.88)  & (79.33,3.29)   & \ldots & (80.66,4.49)   & (86.66,5.31)   \\ \hline
\multicolumn{6}{|c|}{\ldots}                                                                       \\ \hline
23:30         & (139.00,1.63)  & (68.00,2.94)   & \ldots & (69.66,1.69)   & (73.66,1.24)   \\ \hline
\multicolumn{6}{|c|}{Sensor 2}                                                                     \\ \hline
0:00          & (325.91,38.33) & (193.29,54.17) & \ldots & (124.08,33.47) & (129.14,21.29) \\ \hline
0:30          & (300.70,53.96) & (153.54,46.22) & \ldots & (102.95,40.53) & (116.71,23.49) \\ \hline
\multicolumn{6}{|c|}{\ldots}                                                                       \\ \hline
23:30         & (259.20,50.53) & (132.33,39.60) & \ldots & (106.25,31.44) & (106.28,28.49) \\ \hline
\end{tabular}
 }%
\end{center}
\end{table}
%	graph construction
%		adjecency matrix construction
%		all pair shortest path matrix construction
%			why? how?
%sensor map
\section{Sensor Road Network Construction}\label{subsubsec:gc}
Next, we construct a sensor road network which is a weighted directed graph $ G(S,L)$ to indicate the connectivity of the sensors on the road network. We utilize an adjacency matrix $A$ to indicate the neighborhood for our model.
Furthermore, an all-pair shortest path matrix is pre-computed based on the adjacency matrix by using the Floyd-Warshall algorithm~\cite{cormen2001ia} for evaluation.


During sensor road network construction, we sort all sensors of the same direction on the same road by the absolute post-mile field in data records and then connect the sorted neighbor sensors in sequence. Because the connectors of different roads are
missing, we manually add links between sensors and remove wrong links. We compute the shortest path distance based on a road network to evaluate the quality of connectivity, because the Euclidean distance cannot reflect the actual cost from one location to another. 
When there are changes in the sensor metadata, we need to reconstruct the sensor road network and the shortest path matrix. However, the sensor metadata are not updated frequently, so the resource spending on the sensor road network and all-pair shortest path matrix construction is tolerable.


% %%%%%%%%%%%%%%%%%%%%%%%%%%%%%%%%%%%%%%%%%%%%%%%%%%%%%%%%%%
% %Chapter 4: Your work
% \chapter{Traffic Flow Estimation} \label{ch-estimation}
% \input{flow-estimation}

% %%%%%%%%%%%%%%%%%%%%%%%%%%%%%%%%%%%%%%%%%%%%%%%%%%%%%%%%%%
% %Chapter 4: Your work
% \chapter{Anomaly Detection} \label{ch-detection}
% \input{anomaly-detection}

% %%%%%%%%%%%%%%%%%%%%%%%%%%%%%%%%%%%%%%%%%%%%%%%%%%%%%%%%%
% %Chapter 5: Experiments
% \chapter{Experiments} \label{ch-experiment}
% \input{experiment}   

% %%%%%%%%%%%%%%%%%%%%%%%%%%%%%%%%%%%%%%%%%%%%%%%%%%%%%%%%%%
% % Chapter 6: Conclusions
% \chapter{Conclusions}\label{ch-conclusion}
% \input{conclusions}

% %%%%%%%%%%%%%%%%%%%%%%%%%%%%%%%%%%%%%%%%%%%%%%%%%%%%%%%%%%
% %Chapter 7: Future Work
% \chapter{Future Work}\label{ch-future_work}
% \input{futurework}

%%%%%%%%%%%%%%%%%%%%%%%%%%%%%%%%%%%%%%%%%%%%%%%%%%%%%%%%%%

\backmatter
%%%%%%%%%%%%%%%%%%%%%%%%%%%%%%%%%%%%%%%%%%%%%%%%%%%%%%%%%%
%Bibliography
\addcontentsline{toc}{chapter}{\bibname}
\bibliographystyle{ieeetr}%按照引用順序排序%ieeetr
\singlespacing
%若發生reference沒有出現(bibtex編譯時出錯),請先刪除thesis.aux檔案再重新快速編譯一次即可。
%%%%%%%%%%%%%%%%%%%%%%%%%%%%%%%%%%%%%%%%%%%%%%%%%%%%%%%%%(請更改內容)
% input your reference here
\bibliography{papers}
%%%%%%%%%%%%%%%%%%%%%%%%%%%%%%%%%%%%%%%%%%%%%%%%%%%%%%%%%

\appendix
%\chapter{appendix}
%\input{appendix}
\begin{acronym}
 \acro{AID}{Automatic Incident Detection}
 \acro{LBS}{Location-Based Service}
 \acro{GPS}{Global Positioning System}
 \acro{POI}{Point-Of-Interest}
 \acro{GUI}{Graphical User Interface}
 \acro{EWMA}{Exponentially-Weighted Moving Average}
\end{acronym} 

\end{document}
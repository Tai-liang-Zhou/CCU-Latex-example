% 講解語音輸入變得越來越普遍和優點
The types of input of digital devices have become more natural and convenient, and mainly have evolved toward speech in recent years. Talking to personal assistants such as speech interpretation and recognition interface (SIRI) \cite{SIRI} to enable simple tasks accomplished is widely accepted nowadays. Since the popularization of conversational interface applications has been a major trend recently, there is a need for a new application with integrated conversational interfaces.
% 而使用語音或文字輸入最佳的解決方案是chatbot搭配language processing,然後開始介紹chotbot的優點
In particular, the best conversational interface  with voice or text inputs is a chatbot that adopts language processing to handle inputs. Compared to traditional applications, a chatbot can provide a more intuitive user interface to users. Based on natural language processing, a chatbot can interact with humans in natural language, and use the voice recognition to control the device. Unlike humans, chatbots are free from the restriction that serves only a limited number of customers at the same time. Obviously, chatbots can provide several services and receive multiple queries any time at once. Therefore, a chatbot decreases the personnel costs to serve a greater number of customers. Furthermore, a smart chatbot not only learns user’s behaver patterns and made decisions, but also needs to answer correctly to users depending on the conventional context.

% 介紹雲端聊天機器人的缺點
More recently, a chatbot can be easily developed through the existing frameworks such as API.ai \cite{dialogflow}, Wit.ai \cite{wit.ai} and Luis.ai \cite{LUIS}. However, there still exist a lot of challenges in building complex multiple-steps conversations in the processing of developing chatbots. First, a chatbot does not cover such a wide spectrum of capabilities and does not even represent a bot that can chat with people. Most of traditional chatbots do not really have natural language understanding and capability. Actually, these chatbots can only deal with simple keyword matching to inquire the corresponding sentences in the database. Second, the existing frameworks do not provide a sentence generation system for developers. Moreover, it is difficult to define the dialog states and maintain historical conversations.


% 說明現在開發chatbot的缺點
When using those applications, we have to design dialogue management and responses.In particular, we have to create multiple intent components, labeling synonym entities, and tracking states of a chatbot. Most importantly, when designing the dialog management system for chatbots, the understanding of the users’ queries and proper responds need to be carefully handled, because the traditional chatbots are not able to memorize a conversational status that already  generated.


Recent research in the generative adversarial \cite{Deep_Reinforcement_Learning_for_Dialogue_Generation} \cite{SeqGAN} network model has inspired several efforts on conversational systems that apply an encoder to convert an ask sentence to a discrete vector representing its meaning and then generate a response from a decoder. The model generates more diverse, interactive, and non-repetitive responses than the traditional sequence-to-sequence models trained without using generative adversarial networks. A good dialogue model should generate utterances indistinguishable from human dialogues.


% 介紹我自己的演算法,要多寫一點 350 字
% 在這篇論文裡,我們提出建立一個chatbot系統產生句子。我們的model是基於一個自然語言處理的深度學習模型和強化學習模型。
    In this paper, we build a chatbot system for generating sentences. For applying the neural network to our system, each word from the user’s inputs has to be transformed into word embeddings which are numerical representations. Since we adopt Chinese corpus as our dataset, where each Chinese sentence is represented as strings of Chinese characters without explicit delimiters, the procedures of word segmentation and word embedding are required for our language model construction.  Subsequently, the vector of word representation is delivered to the back-end neural network model to generate respond sentences.
    
    
    Our proposed model is based on a sequence natural deep learning model and reinforcement learning model. For the sentence generation, the architecture of the model consists of an encoder and a decoder.  For the encoder, we use the last state from forward and backward bi-LSTM \cite{Bi-LSTM} to be the  history state which  is used to be the initial state  for the decoder. The output of each encoder neuron is used in the attention model to extract the sentence information from the dialogues. We divide the decoder into two kind of condition, the training step and the prediction step. In the training step, the input at each time step of training is the result of the target  which is the corresponding answers from the raw dataset. At the step of prediction, the input is the output at the previous step. We build a generative model from the above architecture. For the discriminator, we use a simple RNN as the basis to get a reward for each word and sentence.

% 簡單提到一下強化學習
Moreover, we utilize the reinforcement learning technique to train our model. Reinforcement learning is a general framework for decision making and can facilitate a response generator to create a sentence that is diverse and understandable. Furthermore, it generates logical responses by maximizing the total future rewards. The higher the reward score, the more fluent the sentence is. To address the discrete dialogue data problem, we apply policy gradient reinforcement learning to back-propagate for updating the parameters used in the model. Therefore, we can achieve improvement in dialogue generation tasks. In the end, the utterances that our model generates could be indistinguishable from human-generated sentences.

% 介紹各章節總結
The rest of the paper is organized as follows. In Section \ref{ch-relatedWork}, we review the important related methods  and the state-of-the-art techniques of chatbots. Section \ref{ch-background} provides the background knowledge. We survey different types of neural network models, attention methods, and generative adversarial networks. In the end, we address the reason why we adopt those methods. In Section \ref{ch-SystemDesign}, we describe the architecture of our chatbot model. For the purpose of generating diverse and understandable sentences, we apply three types of rewards for reinforcement learning. The experiments and evaluation are provided in Section \ref{ch-Experiments}. In Section \ref{ch-Conclusion}, we summarize the main findings and conclude the paper.  In Section 7, we outline the future work for improving our model.